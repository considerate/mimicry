\documentclass[a4paper]{article}
\begin{document}

\section{mimicry}

The mimicry system is intended to avoid modelling global goals explicitly.
%
Instead, each part of the system, referred to as a ``cell'' below, has a set of selfish goals in the hope that these local selfish goals will cause emergent behavior that emulates overarching goals and the ability to solve complex tasks.

\section{Evolution and Self-Reinforcement}

The most basic selfish goal is optimizing for survival and longevity of the cell itself. 
%
However, instead of modelling this goal explicitly each cell is trained replicate its own previous behavior.
%
Given a sufficiently dangerous environment and evolutionary pressures, behavior that is more likely to lead to survival has higher likelihood to be retained since only surviving cells may reproduce.

\section{From Evolution to Social Learning}

To increase the speed at which knowledge is shared in the system each cell is trained not only to replicate their own behavior but also the behavior other cells it determines to be adept at solving the task it is presented with.
%
This shifts the system from being purely evolutionary with knowledge being spread due to sexual reproduction to a memetic one where ideas can be spread more rapidly with transmission of knowledge between different lineages of cells through the means of mimicry.

An initial implementation will provide each cell with the longevity of a collection of neighboring cells.
%
The cell may also inspect the behavior of the neighboring cells and at each time-step the cell randomly samples a neighbor weighted by longevity, whose behavior it will train itself to mimic.
%
If the sampled neighbor died during the time-step the cell instead trains itself to avoid the behavior exhibited by the neighbor.

The system can be further generalized where the cell isn't provided the longevity of the neighboring cells by the system but where each cell instead learns to estimate the fitness of others.
%
This fitness is then used in place of the system-provided longevity score when selecting whose behavior to mimic.

\section{Competitive Environments and Group Dynamics}

When placed in a competitive environment with other cells where the cells compete for the acquirement of resources or where interference competition can utilised to ensure longevity of its own lineage at the expense of other cells, the cell may also use ability to inspect the behavior of its neighbors to better determine its own behavior.

Since each neighbor also has the ability to inspect the behavior of the cell it is not always optimal to exhibit optimal behavior at each time-step.
%
It is of interest to study whether the behavior of the cells will converge to strategies that not only greedily solve the task but to conservative strategies where the cell is less susceptible to competition from its neighbors.

A cell may also be constructed from a collection of sub-cells that collectively determine the behavior of the cell.

\section{Shared Genetics and General-Purpose Modules}

The genetic make-up of the cell, the parameters that determine its behavior, can be shared between the cells in such a way that training the parameters to optimize for one cell also implicitly updates the parameters for all other cells the parameters are shared with.
%
Another form of competition may arise from this sharing of parameters.
%
By making this module of shared parameters better the cell also increases the fitness of its competition resulting in a kind of self-competition where it's not always beneficial to maximize its own ability.

Moreover, if the shared module is just a subset of the make-up of the cells then this module has to adapt to solving the multiple goals of the differing cells.
%
Whether or not this leads to more general-purpose modules or whether the module becomes useless due to excessive competition of the optimization of the parameters could be interesting to look into.

Conversely, if the shared module constitutes the entirety of the parameters of the cell, the cells that share this module may be forced to cooperate if the downsides of competing outweigh the benefits of following a strategy that leads to cooperation with the cells with shared parameters.

\section*{outline?}

\begin{itemize}
\item   Emergent goals from selfish behavior
\item   Mimimicing self-behavior
\item   Dangerous environment
\item   Mimimicing others' behavior
\item   Estimating longevity (fitness) of self and others
\item   Complex goals
\item   Interactions between the agents causing death
\item   Mimimicing other's behavior if good circumstances, avoiding death
    based on others' experiences
\item   Shared genetics
\end{itemize}

\end{document}
